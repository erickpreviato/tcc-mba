% Capitulo 2

% ----------------------------------------------------------
% Revisão Bibliográfica
% ----------------------------------------------------------
\chapter[Revisão Bibliográfica]{Revisão Bibliográfica}
\label{revisao}

\section{Análise de Sentimentos}\label{analise-sentimentos}

A Análise de Sentimentos, ou \textit{Opinion Mining}, é um subcampo do Processamento de Linguagem Natural (PLN) que se dedica à identificação e extração da polaridade (positiva, negativa, neutra) e da subjetividade de textos, classificando-os de acordo com o sentimento expresso \cite{pang2004}. O objetivo principal é determinar a atitude de um autor em relação a um tópico, produto, serviço ou entidade. Essa atitude pode ser um julgamento ou avaliação, um estado emocional, como raiva ou alegria, ou a intenção comunicativa, como a compra de um produto, por exemplo \cite{liu2012}.

A origem da análise de sentimentos pode ser traçada até o início dos anos 2000, impulsionada pelo crescimento exponencial da internet e o consequente aumento na produção de conteúdo textual, como avaliações de produtos, posts em blogs e fóruns de discussão \cite{pang2004}. Embora o estudo da opinião em textos não seja um conceito novo, a abordagem computacional para extrair e resumir essas opiniões ganhou destaque com o advento de técnicas de Aprendizado de Máquina e o volume massivo de dados disponíveis. Os trabalhos pioneiros de Pang, Lee e Vaithyanathan (2002) e Turney (2002) são frequentemente citados como marcos iniciais nesse campo, demonstrando a viabilidade de aplicar algoritmos de aprendizado supervisionado para classificar a polaridade de textos.


\subsection{Tipos e Níveis de Análise de Sentimentos}

A análise de sentimentos pode ser categorizada em diferentes tipos e níveis de granularidade, dependendo do objetivo e da profundidade da análise desejada. A Análise de Sentimentos Baseada em Léxico utiliza léxicos (listas de palavras) pré-definidos, onde cada léxico é associado a uma pontuação de sentimento (positiva, negativa ou neutra). O sentimento de um texto é então calculado com base na soma ou média das pontuações das palavras presentes no texto \cite{hu2004}. Embora simples e eficaz para muitas aplicações, sua principal limitação reside na incapacidade de lidar com ironia, sarcasmo e o contexto em que as palavras são utilizadas.

A Análise de Sentimentos Baseada em Aprendizado de Máquina utiliza algoritmos de aprendizado de máquina, supervisionado ou não supervisionado, para classificar o sentimento de um texto. Modelos são treinados em grandes conjuntos de dados rotulados, no caso supervisionado, para aprender a associar características do texto a uma polaridade de sentimento \cite{medhat2014}.

Em relação aos níveis de granularidade, a análise de sentimentos pode ser realizada em três níveis: nível de documento, nível de sentença e nível de aspecto ou entidade. 

No nível de documento, a classificação do sentimento é realizada em um documento inteiro como positivo, negativo ou neutro. É útil para obter uma visão geral do sentimento em relação a um tópico \cite{pang2004}. 

No nível de sentença, o sentimento de cada frase é analisado individualmente dentro de um documento. Permite identificar nuances e diferentes sentimentos expressos em diferentes partes de um texto \cite{liu2012}. 

Já no nível de aspecto, o mais granular, foca na identificação do sentimento em relação a aspectos ou entidades específicas dentro de um texto \cite{liu2012}. Por exemplo, em uma avaliação de disciplina, pode-se analisar o sentimento em relação ao professor, material didático ou metodologia de avaliação. Este nível é particularmente relevante para o presente trabalho, pois permite identificar sentimentos específicos para as categorias e critérios das avaliações.


\subsection{Aprendizado Supervisionado e Não Supervisionado na Análise de Sentimentos}

A escolha entre abordagens de Aprendizado Supervisionado e Não Supervisionado é fundamental na construção de um modelo de análise de sentimentos robusto e efetivo. A frente de Aprendizado Supervisionado requer um conjunto de dados rotulados, onde cada exemplo de texto já possui uma polaridade de sentimento predefinida, seja ela positiva, negativa ou ainda neutra. O algoritmo aprende a mapear características do texto a partir desses rótulos \cite{patel2024}. A vantagem é que geralmente resulta em modelos com maior precisão quando há dados rotulados de alta qualidade e em quantidade suficiente. Já a desvantagem é com relação à rotulagem de dados, pois é um processo demorado e custoso, especialmente para grandes volumes de texto \cite{medhat2014}. Algoritmos populares na vertente de Aprendizado Supervisionado são: Máquinas de Vetores de Suporte (SVM), \textit{Naive Bayes}, Regressão Logística, Redes Neurais e Árvores de Decisão \cite{shaukat2020}.
    
O Aprendizado Não Supervisionado não requer um conjunto de dados rotulados. Algoritmos desta família buscam padrões e estruturas inerentes aos dados para inferir o sentimento. Métodos baseados em léxico são exemplos de abordagens não supervisionadas, onde a polaridade é inferida a partir de dicionários de sentimentos \cite{hu2004}. Outras técnicas, como modelagem de tópicos, podem ser usadas para identificar temas e associar sentimentos a eles. Uma vantagem é que não exige o esforço de rotulagem manual, tornando-o mais escalável para grandes volumes de dados. Já a desvantagem é que pode apresentar menor precisão em comparação com abordagens supervisionadas e tem dificuldade em capturar nuances de linguagem, como sarcasmo e negação \cite{liu2012}. Algoritmos clássicos na frente de Aprendizado Não Supervisionado são: Análise de Componentes Principais (PCA), \textit{K-Means} (para agrupamento de textos por sentimento implícito) e Modelagem de Tópicos como o \textit{Latent Dirichlet Allocation} (LDA) \cite{islam2018}.

Para o contexto das avaliações de disciplinas, onde a rotulagem manual de 8 mil comentários seria um processo exaustivo, a consideração de abordagens não supervisionadas ou a combinação com técnicas semi-supervisionadas (onde uma pequena parte dos dados é rotulada manualmente e usada para treinar um modelo inicial) pode ser uma estratégia viável. Porém, caso haja possibilidade de utilizar a abordagem supervisionada, ou seja, caso não tenha um custo muito elevado para realizar a rotulagem dos dados, pode-se tornar uma estratégia mais assertiva.


\subsection{Fatores Importantes na Análise de Sentimentos}

A aplicação da análise de sentimentos não é trivial e envolve a consideração de diversos fatores que podem impactar a precisão e a utilidade dos resultados:
\begin{itemize}
    \item Pré-processamento de Texto: Etapa utilizada para limpar e normalizar o texto antes da análise. Inclui a remoção de \textit{stopwords} (palavras comuns como "e", "o", "a"), pontuações, caracteres especiais, lematização (redução de palavras à sua forma base) e \textit{stemming} (redução de palavras ao seu radical) \cite{ganesan2022}.
    \item Tratamento de \textit{Negation}: A presença de negações ("não gostei", "nunca farei") pode inverter o sentido de uma frase e deve ser tratada adequadamente para evitar classificações errôneas \cite{liu2012}.
    \item Linguagem Específica e Gírias: Termos específicos de um domínio (como jargões acadêmicos) e gírias podem não estar presentes em léxicos genéricos e requerem adaptação ou construção de léxicos especializados \cite{pang2004}.
    \item Ironia e Sarcasmo: A detecção de ironia e sarcasmo é um desafio significativo na análise de sentimentos, pois a polaridade literal das palavras é oposta à intenção do autor \cite{liu2012}.
    \item Emojis e Emoticons: Em textos informais, emojis e emoticons são expressivamente utilizados para transmitir sentimentos e devem ser considerados na análise \cite{salgado2024}.
    \item Modelo de Idioma: A análise de sentimentos para a língua portuguesa apresenta desafios específicos devido à sua complexidade gramatical, polissemia e riqueza de sinônimos, exigindo modelos e recursos linguísticos adaptados \cite{souza2020}.
\end{itemize}




\subsection{Ferramentas e Modelos}

Nos últimos anos, a área de análise de sentimentos tem sido significativamente impactada pelo avanço dos modelos de Deep Learning e Transformers. Esses modelos, pré-treinados em grandes volumes de texto, demonstraram capacidades impressionantes de compreensão de linguagem e geração de texto, superando abordagens tradicionais em muitas tarefas de PLN, incluindo a análise de sentimentos \cite{devlin2019}.

É importante ressaltar que a escolha da ferramenta ou modelo depende de diversos fatores, como o volume e a natureza dos dados, a disponibilidade de recursos computacionais, a necessidade de precisão e a granularidade da análise desejada. Para o contexto das avaliações de disciplinas em português, o uso de modelos Transformer pré-treinados em português, como o BERTimbau ou RoBERTa com fine-tuning para o domínio específico, pode oferecer os melhores resultados em termos de precisão e compreensão contextual.

%E: Mostrar uma possível tabela comparativa?
%W: Neste momento, nao eh necessario
%E: OK


\subsection{Trabalhos Relacionados}

%W: Aqui, nesta subsecao, veja se consegue encontrar alguns trabalhos que utilizam modelos de AS no contexto educacional ou ate mesmo similares.
%Para facilitar, veja alguns trabalhos do tipo Survey/Review, como https://arxiv.org/abs/2302.04359, ou papers similares sobre este tema como https://link.springer.com/article/10.1007/s10639-023-11736-2 ou https://www.tandfonline.com/doi/abs/10.1080/17517575.2020.1773542. A partir deles, pode tentar pinçar algumas tecnicas que ja foram descritas nesses trabalhos. Nao é necessário ter muitos trabalhos discutidos (uns 7 ou 8). Discutir de forma bem suscinta e objetiva (ver como os trabalhos acima fazem isso com os outros trabalhos que eles citam).

A análise de sentimentos em avaliações de disciplinas no contexto educacional tem se mostrado uma ferramenta valiosa para compreender a percepção dos estudantes e subsidiar a gestão acadêmica. Alguns estudos recentes exploram essa aplicação, utilizando diferentes técnicas e abordagens para extrair informações de dados textuais.
 
Um estudo recente de \citeonline{koufakou2024} propõe uma arquitetura para mineração de sentimentos a partir de avaliações online de cursos. Para isso, utilizou-se uma abordagem baseada em aprendizado de máquina e modelos como o BERT e RoBERTa, destacando a importância da análise de sentimentos para melhorar a qualidade do ensino e a experiência do aluno. A pesquisa mostra como a identificação de padrões de sentimentos pode fornecer \textit{feedback} construtivo para educadores e administradores de cursos, auxiliando na tomada de decisões estratégicas para uma melhoria contínua.

Na mesma direção, \citeonline{shaik2023} investigaram a aplicação de técnicas de análise de sentimentos para avaliar o \textit{feedback} de estudantes. O trabalho ressalta a capacidade da análise de sentimentos na identificação de pontos fortes e fracos dos programas, permitindo uma resposta mais ágil e direcionada às necessidades dos estudantes.

Já no estudo de \citeonline{grljevic2022}, explorou-se o uso de \textit{big data} e análise de sentimentos para entender a percepção dos alunos sobre o ambiente de aprendizado em instituições de ensino, utilizando informações de um \textit{website} com mais de 8000 avaliações. Embora o foco seja mais amplo, o trabalho aborda a relevância da análise de sentimentos para aprimorar a experiência educacional, fornecendo informações sobre o engajamento e a satisfação dos estudantes.

Esses trabalhos demonstram a relevância e o potencial da análise de sentimentos para aprimorar a gestão educacional, oferecendo uma visão agregada e estratégica a partir de grandes volumes de avaliações, o que corrobora com o objetivo proposto para este trabalho.


%W: Remanejar esta secao para os resultados... futuramente (entao pode suprimir agora deste capítulo)
%E: Ok
% \section{Análise quantitativa}\label{analise-quantitativa}

% Compreender a distribuição das notas e verificar a correlação entre as notas e a análise de sentimentos. Possíveis técnicas: estatísticas descritivas (média, mediana, moda, desvio padrão por pergunta/disciplina/ano), histogramas, boxplots, heatmaps de correlação, dentre outras.

%\section{Análise temporal}\label{analise-temporal}
%As notas estão melhorando ou piorando ao longo do tempo? Utilizar séries temporais?
