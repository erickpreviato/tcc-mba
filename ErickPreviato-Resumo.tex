%% ErickPreviato-Resumo.tex
\setlength{\absparsep}{18pt} % ajusta o espaçamento dos parágrafos do resumo		
\begin{resumo}
	\begin{flushleft} 
        \setlength{\absparsep}{0pt} % ajusta o espaçamento da referência	
        \SingleSpacing 
        \imprimirautorabr~~\textbf{\imprimirtituloresumo}.	\imprimirdata. \pageref{LastPage} p. 
        %Substitua p. por f. quando utilizar oneside em \documentclass
        %\pageref{LastPage} f.
        \imprimirtipotrabalho~-~\imprimirinstituicao, \imprimirlocal, \imprimirdata. 
    \end{flushleft}
    \OnehalfSpacing 			

     % Texto aqui
     A avaliação de disciplinas é um instrumento fundamental para a gestão acadêmica e a melhoria contínua no ensino superior, 
     gerando um vasto volume de dados textuais que torna a análise manual inviável e custosa. Este trabalho propõe a aplicação de técnicas de 
     Processamento de Linguagem Natural (PLN), especificamente a Análise de Sentimentos, para classificar automaticamente o feedback discente. 
     O estudo de caso utilizou uma base de dados contendo cerca de 16.500 avaliações de um instituto da Universidade de São Paulo (USP), coletadas entre 2017 e 2023. 
     A metodologia incluiu o pré-processamento dos dados e uma rotulação híbrida assistida por Large Language Models (LLM). 
     Foram submetidos ao processo de \textit{fine-tuning} modelos baseados na arquitetura \textit{Transformer}, especificamente o BERTimbau (pré-treinado em português) e o 
     XLM-RoBERTa (multilíngue), nas versões Base e Large. Os resultados demonstraram a superioridade dos modelos pré-treinados na língua nativa, 
     com o BERTimbau-Large alcançando o melhor desempenho, registrando uma Acurácia de 90,67\% e um \textit{F1-Score} de 90,37\%. 
     O estudo conclui que a utilização de modelos de linguagem específicos para o português é eficaz para capturar nuances do contexto acadêmico brasileiro, 
     oferecendo uma ferramenta robusta para apoiar a tomada de decisão institucional
 

    \textbf{Palavras-chave}: Análise de sentimentos. Aprendizado de
máquina. Processamento de linguagem natural. BERT. Ensino superior.
\end{resumo}