%% ErickPreviato-Abstract.tex
\begin{resumo}[Abstract]
 \begin{otherlanguage*}{english}
	\begin{flushleft} 
		\setlength{\absparsep}{0pt} % ajusta o espaçamento dos parágrafos do resumo		
 		\SingleSpacing  		\imprimirautorabr~~\textbf{\imprimirtitleabstract}.	\imprimirdata.  \pageref{LastPage} p. 
		%Substitua p. por f. quando utilizar oneside em \documentclass
		%\pageref{LastPage} f.
		\imprimirtipotrabalhoabs~-~\imprimirinstituicao, \imprimirlocal, 	\imprimirdata. 
 	\end{flushleft}
	\OnehalfSpacing 

    % Texto aqui
    Subject evaluation is a fundamental tool for academic management and continuous improvement in higher education, 
	generating a vast volume of textual data that makes manual analysis unfeasible and costly. This work proposes the application of 
	Natural Language Processing (NLP) techniques, specifically Sentiment Analysis, to automatically classify student feedback. 
	The case study used a dataset containing approximately 16,500 evaluations from an institute of the University of São Paulo (USP), 
	collected between 2017 and 2023. The methodology included data preprocessing and a hybrid labeling strategy assisted by Large Language Models (LLM). 
	Transformer-based models, specifically BERTimbau (pre-trained in Portuguese) and XLM-RoBERTa (multilingual), in both Base and Large versions, 
	were submitted to fine-tuning. The results demonstrated the superiority of models pre-trained in the native language, 
	with BERTimbau-Large achieving the best performance, recording an Accuracy of 90.67\% and an F1-Score of 90.37\%. 
	The study concludes that using language models specific to Portuguese is effective in capturing nuances of the Brazilian academic context, 
	offering a robust tool to support institutional decision-making.

    \vspace{\onelineskip}
 
    \noindent 
    \textbf{Keywords}: Sentiment analysis. Natural language processing. Machine learning. BERT. Higher education. 
 \end{otherlanguage*}
\end{resumo}