% Capitulo 1

% ----------------------------------------------------------
% Introdução
% ----------------------------------------------------------
\chapter[Introdução]{Introdução}
\label{Introdução}

\section{Contextualização}\label{contextualizacao}

A era digital em que vivemos tem gerado um volume de dados sem precedentes, transformando a forma como interagimos, consumimos e expressamos nossas opiniões. 
A internet, as redes sociais, fóruns, plataformas de ensino e reviews de produtos tornaram-se repositórios vastos de informações textuais, 
que refletem os pensamentos, sentimentos e percepções dos indivíduos sobre os mais diversos tópicos, produtos e serviços. No contexto educacional, 
a opinião do aluno, expressa através de comentários e realização de avaliações, torna-se uma opção para a melhoria contínua da qualidade do ensino 
e da gestão acadêmica das instituições educacionais \cite{bobo2019}.

A avaliação de disciplinas, em particular, é um processo crucial para instituições de ensino superior, pois permite coletar \textit{feedback} direto 
dos estudantes sobre a qualidade do conteúdo, a metodologia de ensino, a atuação docente e a infraestrutura disponível \cite{romero2020}. 
Em muitos dos casos, essas avaliações são analisadas manualmente: um processo que se torna inviável e ineficiente diante de um grande volume de 
dados gerado ao longo dos anos. A análise individualizada de milhares de comentários, como no caso da base de dados de mais de 8 mil avaliações de 
disciplinas de um instituto da Universidade de São Paulo (USP), consome tempo e recursos significativos, além de ser suscetível a vieses humanos 
e à perda de informações valiosas contidas no conjunto dos dados.

Posto o cenário acima, uma ferramenta proeminente é a Análise de Sentimentos (também conhecida como \textit{Opinion Mining}): um campo da 
Inteligência Artificial e do Processamento de Linguagem Natural (PLN) que visa extrair, identificar e quantificar as polaridades 
emocionais (positiva, negativa ou neutra) expressas em textos \cite{liu2012}. Por meio da aplicação de técnicas computacionais, 
a análise de sentimentos permite transformar dados textuais não estruturados em informações acionáveis, oferecendo uma visão geral e automatizada 
do sentimento predominante em um conjunto de avaliações. Neste caso, o objetivo principal é compreender a atitude, o tom e as emoções expressas 
em um determinado conteúdo textual \cite{liu2020}.

A aplicação da análise de sentimentos no contexto das avaliações de disciplinas de instituições de ensino permite que gestores educacionais 
obtenham uma compreensão do cenário de forma ágil e eficiente. Em vez de examinar cada comentário individualmente, é possível identificar padrões, 
tendências e pontos críticos em um conjunto de avaliações de uma ou mais disciplinas em um dado semestre, por exemplo. Isso facilita 
a tomada de decisões estratégicas, como a revisão de planos de ensino, a capacitação docente, a melhoria de recursos didáticos, 
a otimização das condições de oferta das disciplinas e a evolução de equipamentos tecnológicos, contribuindo assim diretamente para a 
qualidade do ensino e a satisfação dos alunos.

\section{Objetivos}\label{objetivos}

A relevância da análise de sentimentos cresceu conforme a quantidade de informações também evoluiu, tornando-se uma ferramenta para gestão de tomada de decisão. 

Portanto, este trabalho busca explorar a aplicação da análise de sentimentos para automatizar e otimizar o processo de avaliação de disciplinas, 
proporcionando aos gestores uma ferramenta para a compreensão da percepção dos alunos e a consequente melhoria contínua da qualidade educacional.

Em termos de aplicabilidade e prova de conceito, a pesquisa visa validar a escalabilidade da solução por meio de um estudo de caso real utilizando a 
base de dados de um instituto da USP, demonstrando seu potencial para replicação em outras instituições. Os fundamentos teóricos que embasam esses 
objetivos são detalhados no Capítulo \ref{revisao}.

